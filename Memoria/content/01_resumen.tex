\chapter*{Resumen}
Este Trabajo de Fin de Grado se enfoca en el diseño y prototipado de un inversor trifásico dual de 80 kW (2 $\cdot$ 40 kW) y 600 V con control vectorial (FOC) para motores síncronos de imanes permanentes (PMSMs). Este inversor bidireccional busca ser una solución compacta y de alto rendimiento para aplicaciones de tracción eléctrica en vehículos de Formula Student.

En primer lugar se estudia en detalle la máquina eléctrica, abordando su modelo matemático y realizando simulaciones implementando el control vectorial. Posteriormente, se diseña el \textit{hardware} que permitirá integrar este control, y por último, se implementa gran parte del \textit{firmware} necesario para ello. Se validan con rigurosidad los aspectos de la electrónica de potencia para asegurar su funcionamiento conforme a los requisitos establecidos.

El proyecto ha alcanzado una notable densidad de potencia mediante la implementación de tecnologías de vanguardia como los semiconductores de carburo de silicio (SiC). Se plantean líneas de trabajo futuras para mejorar el algoritmo de control, y optimizar el \textit{hardware} y \textit{firmware} del inversor.

Además, este proyecto se concibe como el punto de partida para el desarrollo continuo de un inversor de alto rendimiento. Se espera que los conocimientos adquiridos y los resultados obtenidos sirvan como base sólida para un diseño robusto y eficiente.

\chapter*{Resum}
Aquest Treball de Fi de Grau s'enfoca en el disseny i prototipat d'un inversor trifàsic dual de 80 kW (2 $\cdot$ 40 kW) i 600 V amb control vectorial (FOC) per a motors síncrons d'imants permanents (PMSMs). Aquest inversor bidireccional cerca ser una solució compacta i d'alt rendiment per a aplicacions de tracció elèctrica en vehicles de Formula Student.

En primer lloc s'estudia detalladament la màquina elèctrica, abordant el seu model matemàtic i realitzant simulacions implementant el control vectorial. Posteriorment, es dissenya el \textit{hardware} que permetrà integrar aquest control, i finalment, s'implementa gran part del \textit{firmware} necessari per a això. Es validen amb rigorositat els aspectes de l'electrònica de potència per a assegurar el seu funcionament conforme als requisits establerts.

El projecte ha aconseguit una notable densitat de potència de mitjançant la implementació de tecnologies d'avantguarda com els semiconductors de carbur de silici (SiC). Es plantegen línies de treball futures per a millorar l'algorisme de control, i optimitzar el \textit{hardware} i \textit{firmware} de l'inversor.

A més, aquest projecte es concep com el punt de partida per al desenvolupament continu d'un inversor d'alt rendiment. S'espera que els coneixements adquirits i els resultats obtinguts serveixin com a base sòlida per a un disseny robust i eficient.


\chapter*{Abstract}
This Bachelor's Thesis focuses on the design and prototyping of a dual 80 kW (2 $\cdot$ 40 kW) three-phase inverter with field-oriented control (FOC) for permanent magnet synchronous motors (PMSMs). This bidirectional inverter aims to be a compact and high-performance solution for electric traction applications in Formula Student vehicles.

Firstly, the electric machine is studied in detail, addressing its mathematical model and performing simulations implementing field-oriented control. Subsequently, the hardware is designed to integrate this control, and finally, a significant portion of the firmware necessary for this integration is implemented. The aspects of power electronics are rigorously validated to ensure its operation according to the established requirements.

The project has achieved a notable power density through the implementation of cutting-edge technologies such as silicon carbide (SiC) semiconductors. Future work is proposed to improve the control algorithm and optimize the inverter hardware and firmware.

Furthermore, this project is conceived as the starting point for the continuous development of a high-performance inverter. It is expected that the knowledge acquired and the results obtained will serve as a solid foundation for a robust and efficient design.