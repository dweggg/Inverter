\chapter*{Resumen}
Este Trabajo de Final de Grado se enfoca en el diseño y prototipado de un inversor trifásico dual de 80 kW (2x40 kW) y 600 V con control vectorial (FOC) para PMSMs (motores síncronos de imanes permanentes). Este inversor bidireccional busca ser una solución compacta y de alto rendimiento para aplicaciones de tracción eléctrica en vehículos de Formula Student.

El proyecto tiene la intención de alcanzar altas densidades de potencia, lo que implica la implementación de tecnologías de vanguardia como los semiconductores de carburo de silicio (SiC), el uso de materiales compuestos y la aplicación de técnicas de fabricación aditiva. Se trabajará para alcanzar estos objetivos mediante un diseño que tenga en cuenta la correcta gestión térmica, la disposición eficiente de los componentes, la selección adecuada de cableado y conectores, entre otros aspectos fundamentales.

Además, el código del inversor implementa un lazo de control de par que permite trabajar en la región de debilitamiento de campo, además de optimizar el paso por la zona de baja fuerza contraelectromotriz. El control eficiente de la máquina eléctrica permite extraer al máximo toda su potencia en un rango muy grande de velocidades, llegando de forma segura a sus límites.

\chapter*{Abstract}
This Bachelor's Thesis delves into the design and prototyping of an 80 kW dual three-phase inverter (2x40 kW) operating at 600 V, equipped with field-oriented vector control (FOC) tailored for PMSMs (Permanent Magnet Synchronous Motors). This bidirectional inverter aims to serve as a compact, high-performance solution tailored for electric traction applications in Formula Student vehicles.

The project aims to achieve remarkable power densities, necessitating the incorporation of state-of-the-art technologies like silicon carbide (SiC) semiconductors, along with the utilization of composite materials and additive manufacturing techniques. These endeavors are geared towards realizing a design that meticulously addresses thermal management, efficient component arrangement, optimal wiring and connector selection, among other pivotal considerations.

Moreover, the inverter's code is engineered to feature a torque control loop enabling operation within the field weakening region, while also optimizing trajectory across the low back electromotive force zone. By ensuring efficient control of the electric machine, the system maximizes power extraction across a broad spectrum of speeds, safely pushing its operational limits.