\chapter*{Resumen}
Este Trabajo de Fin de Grado se enfoca en el diseño y prototipado de un inversor trifásico dual de 80 kW (2x40 kW) y 600 V con control vectorial (FOC) para motores síncronos de imanes permanentes (PMSMs). Este inversor bidireccional busca ser una solución compacta y de alto rendimiento para aplicaciones de tracción eléctrica en vehículos de Formula Student.

El proyecto ha alcanzado una notable densidad de potencia de 30 kW/L mediante la implementación de tecnologías de vanguardia como los semiconductores de carburo de silicio (SiC). Se ha trabajado para lograr estos objetivos mediante un diseño que considera la correcta gestión térmica, la disposición eficiente de los componentes y la selección adecuada de conectores, entre otros aspectos fundamentales.

Además, el código del inversor permite controlar de forma completamente independiente dos motores con un solo MCU, e implementa un lazo de control de par que permite operar en la región de debilitamiento de campo. El control eficiente de la máquina eléctrica permite maximizar la extracción de potencia en un amplio rango de velocidades, llegando de forma segura a sus límites operativos.

\chapter*{Resum}
Aquest Treball de Fi de Grau s'enfoca en el disseny i prototipat d'un inversor trifàsic dual de 80 kW (2x40 kW) i 600 V amb control vectorial (FOC) per a motors síncrons d'imants permanents (PMSMs). Aquest inversor bidireccional cerca ser una solució compacta i d'alt rendiment per a aplicacions de tracció elèctrica en vehicles de Formula Student.

El projecte ha aconseguit una notable densitat de potència de 30 kW/L mitjançant la implementació de tecnologies d'avantguarda com els semiconductors de carbur de silici (SiC). S'ha treballat per a aconseguir aquests objectius mitjançant un disseny que considera la correcta gestió tèrmica, la disposició eficient dels components i la selecció adequada de connectors, entre altres aspectes fonamentals.

A més, el codi de l'inversor permet controlar de forma completament independent dos motors amb un sol MCU, i implementa un llaç de control de parell que permet operar a la regió de debilitament de camp. El control eficient de la màquina elèctrica permet maximitzar l'extracció de potència en un ampli rang de velocitats, arribant de manera segura als seus límits operatius.

\chapter*{Abstract}
This Bachelor's Thesis focuses on the design and prototyping of an 80 kW dual three-phase inverter (2x40 kW) operating at 600 V, equipped with field-oriented control (FOC) for Permanent Magnet Synchronous Motors (PMSMs). This bidirectional inverter aims to serve as a compact, high-performance solution for electric traction applications in Formula Student vehicles.

The project has achieved a remarkable power density of 30 kW/L by incorporating state-of-the-art technologies such as silicon carbide (SiC) semiconductors. Efforts have been directed towards realizing a design that meticulously addresses thermal management, efficient component arrangement, and optimal connector selection, among other critical considerations.

Moreover, the inverter's code allows for the completely independent control of two motors with a single MCU, and implements a torque control loop enabling operation within the field weakening region. By ensuring efficient control of the electric machine, the system maximizes power extraction across a broad range of speeds, safely reaching its operational limits.